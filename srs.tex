\documentclass[12pt]{article}
\usepackage{fancyhdr}
\usepackage{amssymb}
\usepackage{tabularx}
\usepackage{hyperref}
\usepackage{geometry}
\geometry{margin=1in}

\author{Sean Allred \\ Molly Domino \\ Josh Kaminsky \\ Matthan Lee}
\date{\today}
\title{Software Requirements Specification}

\hypersetup{
    bookmarks=true,         % show bookmarks bar?
    unicode=false,          % non-Latin characters in Acrobat’s bookmarks
    pdftoolbar=true,        % show Acrobat’s toolbar?
    pdfmenubar=true,        % show Acrobat’s menu?
    pdffitwindow=false,     % window fit to page when opened
    pdfstartview={FitH},    % fits the width of the page to the window
    pdftitle={Software Requirements Specification},    % title
    pdfauthor={No Dice!},     % author
    pdfsubject={SRS},   % subject of the document
    pdfcreator={\LaTeX},   % creator of the document
    pdfproducer={Sean Allred}, % producer of the document
    pdfkeywords={Blackjack}{No Dice!}{SMCM}, % list of keywords
    pdfnewwindow=true,      % links in new window
    colorlinks=true,        % false: boxed links; true: colored links
    linkcolor=red,          % color of internal links
    citecolor=green,        % color of links to bibliography
    filecolor=magenta,      % color of file links
    urlcolor=cyan           % color of external links
}

\pagestyle{fancy}
\rhead{Blackjack\\
Software Requirements Specification\\
COSC 301 -- Software Engineering}
\begin{document}
\maketitle
\thispagestyle{empty}
\newpage
\null
\section{Introduction}

\subsection{Purpose}
The goal of our endeavor is to produce an executable file that will create a
one user blackjack game.  The game should have many functions for the user
including (but not limited to) betting functionality, personalization,
\footnote{Is this really a requirement?} and a graphical user interface.

\subsection{Scope}
We will be creating a file that will play a game of blackjack with hit, split,
and stand functionality. The scope of our project will be between this baseline
and the constraints outlined below:
\begin{itemize}
\item No multiplayer functionality
\item No other game functionality beyond blackjack
\item No animations
\end{itemize}

\subsection{Definitions, Acronyms, and Abbreviations}
\begin{table}[h]
\begin{tabularx}{\linewidth}{|r|X|}
\hline 
\textbf{Term} & Definition \\ \hline 
\textbf{SMCM} & St. Mary's College of Maryland \\ \hline 
\textbf{SME} & Subject Matter Expert \\ \hline 
\textbf{FA} & Functional Analyst \\ \hline 
\textbf{SA} & Solutions Architect \\ \hline 
\textbf{DEV} & Developer \\ \hline 
\textbf{QA} & Quality Assurance \\ \hline 
\textbf{Player} & The user \\ \hline 
\textbf{Dealer} & The computer \\ \hline 
\textbf{Hit} & The move in blackjack wherein the Player draws another Card from
the Deck through the Dealer. \\ \hline 
\textbf{Split} & The move in blackjack wherein the Player's hand is split into
two, where each of the Player's hands take half of the bet. \\ \hline 
\textbf{Stand} & The move in blackjack wherein the Player ceases to draw more
Cards from the Deck, ending the Round. \\ \hline 
\end{tabularx}
\end{table}


\subsection{References}
\begin{itemize}
\item Iteration Plans \\ None yet
\item Development Case \\ None yet
\item Vision \\ None yet
\item Glossary \\ None yet
\item Other \\ Thank you to \href{http://www.github.com}{GitHub.com} for allowing our team to collaborate
online.
\end{itemize}

\subsection{Overview}
\subsubsection{Project Overview}
This \emph{Software Development Plan} provides a description of the purpose,
scope, objectives and deliverables pertaining to the project.

\subsubsection{Project Organization}
\begin{tabular}{|r|c|c|c|c|c|}
\hline 
\textbf{Name} & \textbf{SME} & \textbf{FA} &
\textbf{SA} & \textbf{DEV} & \textbf{QA} \\ 
\hline 
Sean Allred & \null & \checkmark & \checkmark & \checkmark & \checkmark \\ 
\hline 
Molly Domino & \checkmark & \null & \checkmark & \checkmark & \checkmark \\ 
\hline 
Joshua Kaminsky & \checkmark & \checkmark & \checkmark & \null & \checkmark \\ 
\hline 
Matthan Lee & \checkmark & \checkmark & \null & \checkmark & \checkmark \\ 
\hline 
\end{tabular} 

\subsubsection{Management Process}
This project should cost, essentially, only time, and should take about one
semester to complete.  The major phases of this project are defined in
\hyperref[sec:schedule]{sections 4.1 and 4.2}.

\subsubsection{Applicable plans and Guidelines}
Software development team will be following an iterative waterfall model with
two iterations and GUI prototyping, iterating through the design of the logical
structure of the game and the graphical user interface. Tools may include
Microsoft Visual Studio for C\# development\footnote{There also is
consideration of using NetBeans with Java.} and the computers to run these
programming environments. Techniques may include round-table prototyping
sessions and mock-up code demonstrations for the client.

\section{Functional Description}
\subsection{Language and Terminology}
Planned program features will be described as ``necessary'', ``unnecessary'',
or ``pending''; features will also be described as ``intended'' or ``not
intended'', to indicate whether or not they will be implemented.
Including the following language models in the descriptions of various
requirements indicates their category as necessary, unnecessary, planned,
unplanned, or pending categorization.

\begin{table}[!h]
\centering
\begin{tabular}{|r||c|c|}
\hline 
& \textbf{Intended} & \textbf{Unintended} \\ 
\hline 
\textbf{Necessary} & will & should \\ 
\hline 
\textbf{Pending} & may & may not \\ 
\hline 
\textbf{Unnecessary} & can & will not \\ 
\hline 
\end{tabular}
\caption{Terminology}
\end{table}

\textbf{Must} --- Indicates a feature that absolutely must be included in the
software for the software project to be considered finished, and is intended
to be included in the software package.
	
\textbf{Should} --- Indicates a feature that absolutely must be included in the
software for the software project to be considered finished, but is not
intended to be included in the software package.  Concerted effort will be made
to avoid ever using this keyword.
	
\textbf{May} --- Indicates a feature that or may not be necessary, pending
further insight, but is intended to be included in the software package.
	
\textbf{May not} --- Indicates a feature that or may not be necessary, pending
further insight, but is not intended to be included in the software package.
	
\textbf{Will} --- Indicates a feature that is not necessary, but is intended
to be included in the software project.
	
\textbf{Will not} --- Indicates a feature that is not necessary, and is not
intended to be included in the software project.

\subsection{User Interface}
The program must implement a graphical user interface (GUI) as the primary mode
of interactions with the user.
\paragraph{Title}
The program's GUI must be titled for the game being played (i.e. `Blackjack').
\paragraph{Menu}
The GUI must contain a standardized menu bar situated at the top of the
interface with the following items:
\subparagraph{File}
The menu must include the standard `File' menu option that reveals the
following actions:
\begin{itemize}
\item \textbf{Restart} \\ Starts a new game.
\item \textbf{Statistics} \\ Reveals, in a separate window, the Player's
gameplay statistics for the current session: number of wins, number of losses,
largest win, and greatest loss.
\item \textbf{Exit} \\ Upon confirmation, terminates the program.
\end{itemize}
\subparagraph{Help}
The menu must include the standard `Help' menu option that reveals the
following actions:
\begin{itemize}
\item \textbf{About} \\ Reveals various information about the software
including, but not limited to, the title of the program, the version number,
its authors, and its license.
\end{itemize}

\subsection{Graphics}
Program must access and display many visual resources, to include Windows
graphics packages and external image files to represent objects in-game.
\paragraph{Card Display}
Graphics must include an image to represent the back of a playing card (i.e.,
face-down) and the front of a playing card (i.e., face-up).
\paragraph{Deck Display}
Graphics must include an image to represent more than one card in a stack or
deck.
\paragraph{Animation}
Graphics will not include any animations.
\paragraph{Background}
Graphics must include an image to represent the background of the game window
(i.e., the ``table'' on which Blackjack is being played.)
\paragraph{Money Display}
Graphics must include an actively-updated text field in which the Player's
current Funds will be displayed.
\paragraph{Player's Name}
Graphics must include an actively-updated text field in which the Player's Name
is displayed.
\paragraph{Hit Button}
Graphics must include a button for the user to indicate their wish to Hit.
\paragraph{Split Button}
Graphics must include a button for the user to Split.
\paragraph{Stand Button	}
Graphics must include a button for the user to Stand.
\paragraph{Betting Box}
Graphics must include a betting box, which will be located in the upper left
hand of the screen and will display the active bet for the hand. 
\paragraph{Additional effects}
Program may not include any additional effects.

\subsection{Display Screens}
\paragraph{Start-up}
\subparagraph{GUI Initialization}
The screen must appears without any cards dealt.  The player's name and total
starting funds (\$500.00) must be displayed.  Buttons must remain inactive. 
\subparagraph{Session Start}
Program must prompt the user for their user name, and display it on the GUI. 
This prompt will be in the form of a pop-up window or splash screen.
\paragraph{Game over}
Program must display a pop-up window or splash screen when the game ends. It
must display a message telling the user whether they won or lost.  

\section{System Requirements}
\paragraph{Hardware Requirements}
\begin{itemize}
\item 4GB RAM
\item 2.0 GHz CPU (i836)
\end{itemize}
\paragraph{Software Requirements}
\begin{itemize}
\item Microsoft .NET Framework 4
\item Windows 7
\end{itemize}

\section{Interfaces}\label{sec:interfaces}
\subsection{Standalone Program}
The program will neither require nor implement a network interaction of any
type.
\subsection{Use of Windows DLLs}
The program will utilize standard and required class libraries (from Windows 7
and .NET 4), but will require no other resources not specified in System
Requirements.
\subsection{Use of Keyboard and Mouse}
The program must take input via a keyboard and mouse.\footnote{It wouldn't be
hard to add completely keyboard-based functionality to it. It'd be fun.}

\subsection{Performance}
\subsection{Standalone Executable}
Program will be run from a standalone executable (*.exe).
\subsection{System Benchmark}
Program will have a minimal footprint and will operate near-instantaneously on
any machine that meets the system requirements outlined in section 3.1. For
best results, program should be run independently on a dedicated machine by a
trained operator wearing static-resistant clothing.

\section{Delivery}
\subsection{CD-ROM}
Compiled program files will be delivered on a CD-ROM. Users must be able to run
the program by double-clicking or otherwise activating the .exe file on the CD
ROM.

\subsection{Installation}
The program will not contain any installation methods and will not be supported
post-release.  The user will bear the responsibility for accessing the CD-ROM
and activating the executable.

\section{Schedule}\label{sec:schedule}
To be completed.\footnote{Gantt chart?}

\section{Miscellaneous}
\subsection{Specificity}
Blackjack! v. 1.0 by No Dice has been optimized for functionality on machine 4,
lab 165, Physical Sciences Building, St. Mary’s College.

\subsection{Safety}
Blackjack! v. 1.0 has not been proven to be non-carcinogenic or arsenic-free. 
Blackjack! v. 1.0 is pending FDA approval.  The user takes full responsibility
for any damages or injuries incurred by running the program while operating
heavy machinery or during late-term pregnancy.

\end{document}
